\documentclass[12pt]{article}
\usepackage[utf8]{inputenc}
\usepackage{amsmath}
\usepackage{amssymb}
\usepackage{authblk}

\title{The New MININEC (Version 3): A Mini-Numerical Electromagnetic Code}
\author{J. C. Logan and J. W. Rockway}
\affil{Naval Ocean Systems Center San Diego}
\date{September 1986}

\begin{document}
\maketitle
\bibliographystyle{unsrt}

\newcommand{\dd}[1]{\mathrm{d}#1}
\newcommand{\vect}[1]{\bar{#1}}
\newcommand{\Einc}{\vect{E}_{\mathrm{inc}}}

\section{Introduction}
The ``MINI'' Electromagnetics Code, or MININEC, is a method of moments
computer program for analysis of thin wire antennas \cite{r1}. A
Galerkin procedure is applied to an electric field integral equation to
solve for the wire currents following an approach suggested by Wilson
\cite{r2}. This formulation results in an unusually short computer
program suitable for implementation on a microcomputer. Hence, MININEC
is written in a BASIC language compatible with many popular
microcomputers.

MININEC solves for impedance and currents on arbitrarily oriented wires,
including configurations with multiple wire junkctions, in free space
and over a perfectly conducting ground plane. Options include lumped
parameter impedance loading of wires and calculation of near zone and
far zone fields. Both near electric fields and near magnetic fields can
be determined for free space and over a perfectly conducting ground. The
far zone electric fields and radiation pattern (power pattern) can also
be determined for free space and perfectly conducting ground.

Additional radiation pattern options include a Fresnel reflection
coefficient correction to the patterns, for finite conducting grounds
(real earth surface impedance). Up to five changes in surface impedance
due to real ground are allowed in a linear or circular ``cliff'' model.
The cliff may take on any elevation (including zero, i.e., a flat
surface), however, there is no correction for diffraction from cliff
edges. In the case of a circular cliff model, the first media may
include a correction for the surface impedance of a densely spaced,
buried, radial wire ground screen.

The first version of MININEC given by NOSC TD 516 \cite{r1}, calculated
currents and radiation patterns for wire antennas in free space and over
a perfectly conducting ground plane. Wires attached to ground were
required to intersect at a right angle and could not be impedance loaded
at the connection point. Subsequent revisions corrected these
shortcomings culminating in version 2 of MININEC(2), given by Li, et al.
\cite{r3}. All previous versions of MININEC require user specification
of wire end connections. However, MININEC(3) determines connection
information for itself from user defined wire end coordinates.
MININEC(3) also displays the currents wire by wire, and at all wire
ends, including wire junctions. MININEC(3) features an improved, faster
solution routine and has been completely restructured using a more
modular programming style, including the use of helpful comment
statements.

\subsection{Background}
The Numerical Electromagnetics Code (NEC) found in reference \cite{r4}
is the most adavanced computer code available for the analysis of thin
wire antennas. It is a highly user-oriented computer code offering a
comprehensive capability for analysis of the interaction of
electromagnetic waves with conducting structures. The program is based
on the numerical solution of integral equations for the currents induced
on the structure by an exciting field.

NEC combines an integral equation for smooth surfaces with one for wires
to provide convenient and accurate modeling for a wide range of
applications. A NEC model may include nonradiating networks and
transmission lines, perfect and imperfect conductors, lumped element
loading, and ground planes. The ground planes ma be perfectly or
imperfectly conducting. Excitation may be via applied voltage source or
incident plane wave. The output may include induced currents and
charges, near or far zone electric or magnetic fields, and impedance or
admittance. Many other commonly used parameters such as gain and
directivity, power budget, and antenna to antenna coupling are also
available.

NEC is a powerful tool for many engineering applications. It is ideal
for modeling co-site antenna environments in which the interaction
between antenna and environment cannot be ignored. In many problems,
however, the extensive full capability of NEC is not really required
because the antenna and its environment are not very complex or the
information sought requires only a simplified model. In addition, NEC
requires the support of and access to a large main-frame computer
system. These computer systems are expensive and not always readily
available at remote field activities. Even when the computer facilities
are available, heavy demand usage may result in slow turnaround, even
for relatively simple or small NEC runs. One viable solution is a
``stripped down'' version of NEC that would retain only the basic
solution and the most frequently used options and which could be
implemented on a mini- or microcomputer with an advanced FORTRAN
language capability. MININEC(3) offers many of the required NEC options,
but makes use of a BASIC language that is compatible with many popular
microcomputers. MININEC(3) is only suitable for small problems less than
75 unknowns and 10 wires, depending on the computer memory and BASIC
compiler.

\subsection{Computer Requirements}
Occasionally a technology develops which is destined to produce
significant changes in the way people think and conduct their business.
For many decades, scientists and engineers struggled with unmanageable
equations and data using trial and error techniques, employing
logarithmic tables and inadequate slide rule calculations. Then came the
digital computer.

In the 1950s and 60s, physically large and expensive computing machines
that were relatively slow, with limited capability compared to today's
standards, became available to few. At first, stored programs were
accessible through direct connection of individual terminals a short
distance away. The revolution had begun.

In the 70s, technologists rushed to convert proven algorithms into
computer programs or to develop new algorithms suitable for efficient
computer programming for use as analysis and synthesis tools by the
scientific community. These tools, for the most part, required the
support of large central machines. Meanwhile, slide rules were being
replaced by hand-held calculators with trigonometric functions, some of
which could be programmed for simple repetitive algorithms.

Today, large central processing systems are being supplemented with
small powerful mini- and microcomputers. The development of the low cost
microprocessor chip means that computers with capabilities that equal or
exceed those of the earlier main frame machines of the 60s are now
available in compact size. Sizes range from suitcase, or desktop,
machines the microcomputer to file cabinet machines the minicomputer
that [unreadable] expanded or configured to meet specialized needs. The
microcomputer is becoming more and more affordable as a personal
computing tool. The microcomputer, or ``home computer'', is emerging as
today's most important engineering and scientific tool, allowing
widespread networking. Anyone with a microcomputer or terminal with an
acoustic coupler and telephone has access to a wide variety of computing
facilities around the country, as well as an almost limitless source of
information.

MININEC has been written with the microcomputer in mind. But, it can
also be implemented on mini- or larger computers that have the BASIC
language capability. However, some changes in the program may be
required. Programming hs been kept simple, with few machine-dependent
program statements, so that it will be compatible with most BASIC
languages.

NEC is suitable for both small and large numeric models. The upper limit
is determined by the cost factors and memory size of the mainframe on
which it resides. A model containing up to 2000 unknowns (segments)
seems to be the practical upper limit. On the other hand, MININEC is
suitable only for small problems. The upper limit is determined by the
memory size and speed of the microcomputer employed. Practical limits
seem to be 30 to 40 unknowns (current pulses) when using interpreter
BASIC, due to the time required to obtain a solution. However, if one is
willing to wait an hour or more for the solution, a model with 65 to 75
unknowns is possible. Serious antenna modeling requires the use of a
BASIC compiler. In addition, a math co-processor board is recommended.
Present microcomputer memory size limits MININEC to models with less
than 100 unknowns. For problems of 100 or more unknowns, a mainframe is
recommended, and in that case, the use of NEC is the natural choice.

\section{The Theory of MININEC}
The MININEC program is based on the numerical solution of an integral
equation representation of the electric fields. Discussion of similar
formlations can be found elsewhere, for example, see Harrington
\cite{r5}. The real advantage is that the solution techniques as
implemented in MININEC results in a relatively compact (i.e. short)
computer code. The discussion that follows in this section is condensed
from reference \cite{r2}.

\subsection{The Electric Field Integral Equation and its Solution}
It has become customary in solving wire antenna problems to make several
assumptions which are valid for thin wires. They are that the wire
radius $a$, is very small with respect to the wavelength and the wire
length. Because it is necessary to subdivide wires into short segments,
the radius is assumed small with respect to the segment lengths as well,
so that the currents can be assumed to be axially directed; i.e., there
are no azimuthal components of current.

Figure~\eqref{fig1} gives the geometry of a typical, arbitrarily oriented
wire. Assume that the wire is straight, even though the theory applies
equally to bent configurations. The same wire is also shown broken into
segments or subsections.

In equations \eqref{eq1}, \eqref{eq2}, and \eqref{eq3} below, the vector and
scalar potentials are given by

\begin{equation}
\vect{A} = \frac{\mu}{4\pi} \int_c I(s)\hat{s}(s)k(s-s^\prime)\dd{s}
\label{eq1}
\end{equation}

\begin{equation}
\Phi = \frac{1}{4\pi\varepsilon} \int_c q(s)k(s-s^\prime)\dd{s}
\label{eq2}
\end{equation}

where
\[ k(s-s\prime) = \frac{1}{2\pi}\int_{-\pi}^{\pi} \frac{e^{-jkr}}{r}\dd{\Phi}
\]

\[ r = ((s-s^\prime)^2 + 4a^2\sin^2\frac{\Phi}{2})^{\frac{1}{2}}
\]

and the linear charge density (via the continuity equation) is
\begin{equation}
q(s) = \frac{-1}{j\omega}\frac{\dd{I}}{\dd{s}}
\label{eq3}
\end{equation}

The kernel $k$ becomes the ``exact kernel'' when
$\vect{r}\rightarrow\vect{r^\prime}$ on $c$, but can be accurately
replaced by the ``reduced kernel,''
$k_0 = e^{-jkr}/r$, $|\vect{r} - \vect{r}(s)|^2+a^2(s)^{1/2}$
for $|\vect{r}-\vect{r^\prime}| \gg a$.

The integral equation relating the incident field, $\Einc$, and the
vector and scalar potentials is

\begin{equation}
-\Einc\cdot\hat{s}\times
-j\omega\hat{A}\cdot\hat{s}-\hat{s}\cdot\nabla\Phi
\quad .
\label{eq4}
\end{equation}

Equation \eqref{eq4}, above, is solved in MININEC by using the following
procedure.

The wires are divided into equal segments, and, as shown in
Figure~\ref{fig1}, the vectors $\vect{r}_n, n=0, 1, \ldots N+1$ are
defined, with respect to the global coordinate origin, $0^n$. The unit
vectors parallel to the wire axis for each segment shown are defined as

\begin{equation}
\hat{s}_{n+1/2} = \frac{\vect{r}_{n+1}
- \vect{r}_n}{|\vect{r}_{n+1}-\vect{r}_n|}
\quad .
\label{eq5}
\end{equation}

Pulse testing and pulse expansion functions used in MININEC are defined as
\begin{equation}
P_n(s) = \left\{
\begin{array}{ll}
1, & s_{n-1/2} < s < s_{n+1/2} \\
0, & \mathrm{otherwise}        \\
\end{array}\right.
\label{eq6}
\end{equation}

where the points $s_{n\pm1/2}$ designate segment midpoints,

\begin{equation}
s_{n+1/2} = \frac{s_{n+1} + s_n}{2}
\label{eq7}
\end{equation}

or in terms of the global coordinates,

\begin{equation}
\hat{r}_{n+1/2} = \frac{\hat{r}_{n+1} + \hat{r}_n}{2}
\quad.
\label{eq8}
\end{equation}

It is assumed that the components of the vectors $\Einc$
and $\vect{A}$ in equation \eqref{eq4} are sufficiently smooth over each
segment that their respective values on each segment may be replaced by
those taken at the point $s_m$. The pulse functions of \eqref{eq6} are
then used as testing functions on \eqref{eq4}, resulting in

\begin{equation}
\begin{aligned}
\Einc(s_m)&\cdot
\left[\left(\frac{s_m-s_{m-1}}{2}\right)\hat{s}_{m-1/2}
+ \left(\frac{s_{m+1} - s_m}{2}\right)\hat{s}_{m+1/2}\right] = \\
j\omega\vect{A}(s_m)&\cdot
\left[\left(\frac{s_m-s_{m-1}}{2}\right)\hat{s}_{m-1/2}
+ \left(\frac{s_{m+1}-s_m}{2}\right)\hat{s}_{m+1/2}\right] + \\
\Phi(s_{m+1/2}) &- \Phi(s_{m-1/2}) \\
\end{aligned}
\label{eq9}
\end{equation}

The vector quantities in brackets are simply
$(\vect{r}_{m+1/2} - \vect{r}_{m-1/2})$,
so \eqref{eq9} can be written as

\begin{equation}
\begin{gathered}
\Einc(s_m)\cdot(\vect{r}_{m+1/2}-\vect{r}_{m-1/2}) = \\
j\omega\vect{A}(s_m)\cdot(\vect{r}_{m+1/2}-\vect{r}_{m-1/2})
+\Phi(s_{m+1/2}) - \Phi(s_{m-1/2})
\quad.
\end{gathered}
\label{eq10}
\end{equation}

The currents are expanded in pulses centered at the junctions of
adjacent segments as illustrated in Figure~\ref{fig2a}. Note that pulses
are omitted from the wire ends. This is equivalent to placing a half
pulse of zero amplitude at each end, thus imposing the boundary
condition for zero current at unattached wire ends. The current
expansion can be written as

\begin{equation}
I(s) = \sum_{n=1}^{N} I_n P_n (s)
\quad.
\label{eq11}
\end{equation}

A difference approximation is applied to equation \eqref{eq3} to compute
the charge. Thus, as shown in Figure~\ref{fig2b}, the charge can be
represented as pulses displaced from the current pulses by a half pulse
width.

Substituting \eqref{eq11} into \eqref{eq10} produces a system of
equations that can be expressed in matrix form. Each matrix element,
$Z_{mn}$, associated with the $n$-th current and the $s_m$ observation
point involves scalar and vector potential terms with integrals of the
form

\begin{equation}
\Phi_{m,u,v} = \int_{s_u}^{s_v} k(s_m - s^\prime)\dd{s^\prime}
\label{eq12}
\end{equation}

where

\begin{equation}
k(s-s^\prime) = \frac{1}{2\pi}\int_{-\pi}^\pi\frac{e^{-jkr}m}{r_m}\dd{\Phi}
\label{eq13}
\end{equation}

and

\begin{equation}
r_m = \left((s_m - s^\prime)^2 + 4a^2\sin^2\frac{\Phi}{2}\right)^{1/2}
\quad.
\label{eq14}
\end{equation}

Equation \eqref{eq12} does not lend itself to straightforward
integration because of the singularity at $r=0$. The $1/r$ can be
subtracted from the integrand and then added as a separate term to yield

\begin{equation}
k(s-s^\prime) = \frac{1}{2\pi}\int_{-\pi}^\pi \frac{\dd{\Phi}}{r_m}
+\frac{1}{2\pi}\int_{-\pi}^\pi\frac{e^{-jkr}m-1}{r_m}\dd{\Phi}
\quad.
\label{eq15}
\end{equation}

The first term of \eqref{eq15} can be rewritten as an elliptic integral
of the first kind (reference~\cite{r6}).

\begin{equation}
\frac{\beta}{\pi a}F\left(\frac{\pi}{2}, \beta\right) =
\frac{1}{2\pi}\int_{-\pi}^\pi\frac{\dd{\Phi}}{r_m}
\label{eq16}
\end{equation}

where
\[ \beta=\frac{2a}{\left(s_m-s^\prime)^2+4a^2[\right]^{1/2}}
\]

$F(\frac{\pi}{2}$ has an approximation (reference~\ref{r6}).

\begin{equation}
\begin{aligned}
F\left(\frac{\pi}{2}, \beta\right) \cong\ 
& [a_0m + a_1m + a_2m^2 + a_3m^3] \cdot     \\
& [b_0 + b1_m + b_2m^2 + b_3m^3]\ln(1/m)    \\
\end{aligned}
\label{eq17}
\end{equation}

where

\[
m = 1 - \beta^2 = \frac{(s_m - s^\prime)^2}{(s_m - s^\prime)^2 + 4a^2}
\]

and

\[
\begin{array}{lrl}
a_0 = & 1.38629\ 436112 & b_0 = .5            \\
a_1 = &  .09666\ 344259 & b_1 = .12498\ 59397  \\
a_2 = &  .03590\ 092383 & b_2 = .06880\ 248576 \\
a_3 = &  .03742\ 563713 & b_3 = .03328\ 355346 \\
a_4 = &  .01451\ 196212 & b_4 = .00441\ 787012 \\
\end{array}
\]

Thus

\begin{equation}
\frac{\beta}{\pi a}F\left(\frac{\pi}{2},\beta\right)
\overrightarrow{s\rightarrow s^\prime}
-\frac{1}{\pi a}\ln\left[\frac{|s_m - s^\prime|}{8a}\right]
\label{eq18}
\end{equation}

and this singularity is also subtracted from $k(s_m-s^\prime)$.

Thus

\begin{equation}
\begin{gathered}
k(s_m - s^\prime) = -\frac{1}{\pi a}\ln \left[\frac{|s_m - s^\prime|}{8a}\right]
+\frac{\beta F(\frac{\pi}{2}, \beta)
      + \ln\left[\frac{|s_m - s^\prime|}{8a}\right]}{\pi a}  \\
+ \frac{1}{2\pi}\int_{-\pi}^\pi\frac{e^{-jkr}-1}{r}\dd{\Phi} \\
\end{gathered}
\label{eq19}
\end{equation}

This equation is substituted into equation~\eqref{eq12} and written as

\begin{equation}
\int_{s_u}^{s_v} k(s-s^\prime)\dd{s^\prime} = I_1 + I_2 + I_3
\quad.
\label{eq20}
\end{equation}

$I_1$, $I_2$, and $I_3$ are defined as

\begin{equation}
\begin{aligned}
I_1 & = -\frac{1}{\pi a}\int_{s_u}^{s_v}
        \ln\left[\frac{|s-s^\prime|}{8a}\right] \dd{s^\prime}
    & = \left.\frac{8}{\pi} u(1-\ln|u|)\right|_{u_1}^{u_2}
\end{aligned}
\label{eq21}
\end{equation}

where

\[ u_1 = \frac{s_u - s}{8a} \mbox{and} u_2 = \frac{s_v - s}{8a}
\]

Similarly,

\begin{equation}
I_2 = \int_{s_u}^{s_v}
\frac{\beta F\left(\frac{\pi}{2}, \beta\right)
+ \ln\frac{|s-s^\prime|}{8a}}{\pi a}\dd{s^\prime}
\label{eq22}
\end{equation}

This integral has a well behaved integrand and can be integrated
numerically. The integration is broken up into two integrals over the
ranges $(s_u, s)$ and $(s, s_v)$ for best accuracy. Gaussian quadrature
is used for the numerical integration (reference~\cite{r7}). The number
of points used in the integration routine is automatically selected by
consideration of the pulse accuracy required for the source to
observation distance. The final integral is

\begin{equation}
I_3 = \frac{1}{2\pi}\int_{s_u}^{s_v}\int_{-\pi}^\pi\frac{e^{-jkr}-1}{r}\dd{\Phi}
\quad.
\label{eq23}
\end{equation}

The integrand is nonsingular and can be integrated numerically. To
obviate the need for double integration, it is convenient to approximate
the integral by replacing $r$ by a reduced kernel approximation of
equation~\eqref{eq14}.

Thus

\begin{equation}
I_3 = \int_{s_u}^{s_v}\frac{e^{-jkr_a}-1}{r_a}\dd{s^\prime}
\label{eq24}
\end{equation}

where

\[
r_a = \sqrt{(s_v - s^\prime) + a^2}
\quad.
\]

The integral can be integrated numerically by the same procedure as for
$I_2$.

Thus, equation~\eqref{eq12} with its singularity problem is evaluated by
adding $I_1$ of equation~\eqref{eq21}, $I_2$ of equation~\eqref{eq22}, and
$I_3$ of equation~\eqref{eq24}.

This approach to evaluate~\eqref{eq12} is accurate for a wide range of
wire radii but breaks down when the radius becomes very small. For very
small radii, equation~\eqref{eq12} may be expressed as a single integral
and evaluated using two terms of a Maclaurin series, after Harrington
(reference~\cite{r5}). This approximation for the $\Phi$ terms is:

\begin{equation}
\begin{array}{ll}
\Phi = \frac{1}{2\pi\delta s}\ln\left(\frac{\delta s}{a}\right)
-j\frac{k}{4\pi}                   & \mbox{for } m=n          \\
\Phi = \frac{e^{-jkr_m}}{4\pi r_m} & \mbox{for } m\ne n \quad.\\
\end{array}
\label{eq25}
\end{equation}

Figure~\ref{fig3} demonstrates the range and validity with and without
the small radius correction. Without the correction, MININEC gives
acceptable answers for wire radii between $10^{-2}$ and $10^{-5}$ wave
lengths. Note that MININEC is within 10\% or better of the data
published by King (references~\cite{r8} and~\cite{r9}), for radii
between $10^{-3}$ and $10^{-2}$ wave lengths. The small radius
correction provides correct results for radii of $10^{-4}$ wave lengths
or smaller. In MININEC, the switch to the small radius approximations
occurs automatically for radii of $10^{-4}$ and smaller.

By substitution, the matrix equation to be solved is

\begin{equation}
[Z_{mn}] = [I_n] = [V_m]
\label{eq26}
\end{equation}

where

\begin{equation}
\begin{gathered}
[Z_{mn}] = \left.\frac{-1}{4\pi j\omega\varepsilon}\right[
k^2(\vect{r}_{m+1/2} - \vect{r}_{m-1/2}) \cdot
(\hat{s}_{n+1/2}\Phi_{m,n,n+1/2} + \hat{s}_{n-1/2}\Phi_{m,n-1/2,n}) \\
\left.
-\frac{\Phi_{m+1/2,n,n+1}}{s_{n+1}-s_n}+\frac{\Phi_{m+1/2,n-1,n}}{s_n-s_{n-1}}
+\frac{\Phi_{m+1/2,n,n+1}}{s_{n+1}-s_n}-\frac{\Phi_{m+1/2,n-1,n}}{s_n-s_{n-1}}
\right]\\
\end{gathered}
\label{eq27}
\end{equation}

and
\begin{equation}
V_m = \Einc(s_m)\cdot(\vect{r}_{m+1/2}-\vect{r}_{m-1/2})
\quad.
\label{eq28}
\end{equation}

$[Z_{mn}]$ is a square matrix and $[I_n]$ and $[V_m]$ are column
matrices with $n=1,2,\ldots N$ and $m=1,2,\cdot N$ total unknowns ($N$ is
the total number of current pulses). The extension of these equations to
two or more coupled wires follows the same line of development and will
not be covered here.

The column vector $[V_m]$ represents an applied voltage that
superimposes a constant tangential electric field along the wire for a
distance of one segment length centered conincident with the location of
the current pulses. Hence, for a transmitting antenna, all elements of
$V_m]$ are set to zero except for the element(s) corresponding to the
segment(s) located at the desired feed point(s). For an incident plane
wave, all elements of $[V_m]$ must be assigned a value depending on the
strength, polarization (or orientation), and angle of incidences of the
plane wave. The applied voltage source (transmit case), however, is the
only ready-made, or programmed, option in MININEC.

As stated above, the $[Z_{mn}]$ matrix in equation~\eqref{eq26} is
filled by the evaluation of an elliptic integral and use of Gaussian
quadrature for numerical integration. The solution of~\eqref{eq26} can
be accomplished by using any one of a number of standard matrix solution
techniques. MININEC(3) uses a triangular decomposition (LU
decomposition) with the Gauss elimination procedure with partial
pivoting (reference~\cite{r7}).

\section{Wire Junctions}
The theory developed thus far for straight wires is equally applicable
to bent wires. However, for coding simplicity in MININEC, bent wires are
treated in the same way as the junctions of multiple numbers of wires.
That is, a bend in an otherwise straight wire is treated as the junction
between two straight wires.

It has been generally accepted that the currents at junctions of thin
wires conform to Kirchoff's current law (reference~\cite{r10}). Rather
than explicitly enforcing this condition in MININEC, an overlapping
segment scheme (reference~\cite{r11}) is employed at junctions of two or
more wires. A detailed discussion of this approach, including arguments
for validity, appears in both references~\cite{r8} and~\cite{r9}. Only
those aspects essential to the use of MININEC are discussed here.

\end{document}
