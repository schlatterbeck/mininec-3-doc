\documentclass[12pt]{article}
\usepackage[utf8]{inputenc}
\usepackage{amsmath}
\usepackage{amssymb}
\usepackage{authblk}

\title{The New MININEC (Version 3): A Mini-Numerical Electromagnetic Code}
\author{J. C. Logan and J. W. Rockway}
\affil{Naval Ocean Systems Center San Diego}
\date{September 1986}

\begin{document}
\maketitle
\bibliographystyle{unsrt}

\newcommand{\dd}[1]{\mathrm{d}#1}
\newcommand{\vect}[1]{\bar{#1}}
\newcommand{\Einc}{\vect{E}_{\mathrm{inc}}}
\newcommand{\Hplus}{\stackrel{H}{+}}
\newcommand{\Aplus}{\stackrel{A}{+}}
\newcommand{\Pin}{P_{\mbox{IN}}}
\newcommand{\ave}[1]{#1_{\mbox{ave}}}
\newcommand{\peak}[1]{#1_{\mbox{peak}}}
\newcommand{\bounce}[1]{#1_{\mbox{bounce}}}

\section{Introduction}
The ``MINI'' Electromagnetics Code, or MININEC, is a method of moments
computer program for analysis of thin wire antennas \cite{r1}. A
Galerkin procedure is applied to an electric field integral equation to
solve for the wire currents following an approach suggested by Wilson
\cite{r2}. This formulation results in an unusually short computer
program suitable for implementation on a microcomputer. Hence, MININEC
is written in a BASIC language compatible with many popular
microcomputers.

MININEC solves for impedance and currents on arbitrarily oriented wires,
including configurations with multiple wire junkctions, in free space
and over a perfectly conducting ground plane. Options include lumped
parameter impedance loading of wires and calculation of near zone and
far zone fields. Both near electric fields and near magnetic fields can
be determined for free space and over a perfectly conducting ground. The
far zone electric fields and radiation pattern (power pattern) can also
be determined for free space and perfectly conducting ground.

Additional radiation pattern options include a Fresnel reflection
coefficient correction to the patterns, for finite conducting grounds
(real earth surface impedance). Up to five changes in surface impedance
due to real ground are allowed in a linear or circular ``cliff'' model.
The cliff may take on any elevation (including zero, i.e., a flat
surface), however, there is no correction for diffraction from cliff
edges. In the case of a circular cliff model, the first media may
include a correction for the surface impedance of a densely spaced,
buried, radial wire ground screen.

The first version of MININEC given by NOSC TD 516 \cite{r1}, calculated
currents and radiation patterns for wire antennas in free space and over
a perfectly conducting ground plane. Wires attached to ground were
required to intersect at a right angle and could not be impedance loaded
at the connection point. Subsequent revisions corrected these
shortcomings culminating in version 2 of MININEC(2), given by Li, et al.
\cite{r3}. All previous versions of MININEC require user specification
of wire end connections. However, MININEC(3) determines connection
information for itself from user defined wire end coordinates.
MININEC(3) also displays the currents wire by wire, and at all wire
ends, including wire junctions. MININEC(3) features an improved, faster
solution routine and has been completely restructured using a more
modular programming style, including the use of helpful comment
statements.

\subsection{Background}
The Numerical Electromagnetics Code (NEC) found in reference \cite{r4}
is the most adavanced computer code available for the analysis of thin
wire antennas. It is a highly user-oriented computer code offering a
comprehensive capability for analysis of the interaction of
electromagnetic waves with conducting structures. The program is based
on the numerical solution of integral equations for the currents induced
on the structure by an exciting field.

NEC combines an integral equation for smooth surfaces with one for wires
to provide convenient and accurate modeling for a wide range of
applications. A NEC model may include nonradiating networks and
transmission lines, perfect and imperfect conductors, lumped element
loading, and ground planes. The ground planes ma be perfectly or
imperfectly conducting. Excitation may be via applied voltage source or
incident plane wave. The output may include induced currents and
charges, near or far zone electric or magnetic fields, and impedance or
admittance. Many other commonly used parameters such as gain and
directivity, power budget, and antenna to antenna coupling are also
available.

NEC is a powerful tool for many engineering applications. It is ideal
for modeling co-site antenna environments in which the interaction
between antenna and environment cannot be ignored. In many problems,
however, the extensive full capability of NEC is not really required
because the antenna and its environment are not very complex or the
information sought requires only a simplified model. In addition, NEC
requires the support of and access to a large main-frame computer
system. These computer systems are expensive and not always readily
available at remote field activities. Even when the computer facilities
are available, heavy demand usage may result in slow turnaround, even
for relatively simple or small NEC runs. One viable solution is a
``stripped down'' version of NEC that would retain only the basic
solution and the most frequently used options and which could be
implemented on a mini- or microcomputer with an advanced FORTRAN
language capability. MININEC(3) offers many of the required NEC options,
but makes use of a BASIC language that is compatible with many popular
microcomputers. MININEC(3) is only suitable for small problems less than
75 unknowns and 10 wires, depending on the computer memory and BASIC
compiler.

\subsection{Computer Requirements}
Occasionally a technology develops which is destined to produce
significant changes in the way people think and conduct their business.
For many decades, scientists and engineers struggled with unmanageable
equations and data using trial and error techniques, employing
logarithmic tables and inadequate slide rule calculations. Then came the
digital computer.

In the 1950s and 60s, physically large and expensive computing machines
that were relatively slow, with limited capability compared to today's
standards, became available to few. At first, stored programs were
accessible through direct connection of individual terminals a short
distance away. The revolution had begun.

In the 70s, technologists rushed to convert proven algorithms into
computer programs or to develop new algorithms suitable for efficient
computer programming for use as analysis and synthesis tools by the
scientific community. These tools, for the most part, required the
support of large central machines. Meanwhile, slide rules were being
replaced by hand-held calculators with trigonometric functions, some of
which could be programmed for simple repetitive algorithms.

Today, large central processing systems are being supplemented with
small powerful mini- and microcomputers. The development of the low cost
microprocessor chip means that computers with capabilities that equal or
exceed those of the earlier main frame machines of the 60s are now
available in compact size. Sizes range from suitcase, or desktop,
machines the microcomputer to file cabinet machines the minicomputer
that [unreadable] expanded or configured to meet specialized needs. The
microcomputer is becoming more and more affordable as a personal
computing tool. The microcomputer, or ``home computer'', is emerging as
today's most important engineering and scientific tool, allowing
widespread networking. Anyone with a microcomputer or terminal with an
acoustic coupler and telephone has access to a wide variety of computing
facilities around the country, as well as an almost limitless source of
information.

MININEC has been written with the microcomputer in mind. But, it can
also be implemented on mini- or larger computers that have the BASIC
language capability. However, some changes in the program may be
required. Programming hs been kept simple, with few machine-dependent
program statements, so that it will be compatible with most BASIC
languages.

NEC is suitable for both small and large numeric models. The upper limit
is determined by the cost factors and memory size of the mainframe on
which it resides. A model containing up to 2000 unknowns (segments)
seems to be the practical upper limit. On the other hand, MININEC is
suitable only for small problems. The upper limit is determined by the
memory size and speed of the microcomputer employed. Practical limits
seem to be 30 to 40 unknowns (current pulses) when using interpreter
BASIC, due to the time required to obtain a solution. However, if one is
willing to wait an hour or more for the solution, a model with 65 to 75
unknowns is possible. Serious antenna modeling requires the use of a
BASIC compiler. In addition, a math co-processor board is recommended.
Present microcomputer memory size limits MININEC to models with less
than 100 unknowns. For problems of 100 or more unknowns, a mainframe is
recommended, and in that case, the use of NEC is the natural choice.

\section{The Theory of MININEC}
The MININEC program is based on the numerical solution of an integral
equation representation of the electric fields. Discussion of similar
formlations can be found elsewhere, for example, see Harrington
\cite{r5}. The real advantage is that the solution techniques as
implemented in MININEC results in a relatively compact (i.e. short)
computer code. The discussion that follows in this section is condensed
from reference \cite{r2}.

\subsection{The Electric Field Integral Equation and its Solution}
It has become customary in solving wire antenna problems to make several
assumptions which are valid for thin wires. They are that the wire
radius $a$, is very small with respect to the wavelength and the wire
length. Because it is necessary to subdivide wires into short segments,
the radius is assumed small with respect to the segment lengths as well,
so that the currents can be assumed to be axially directed; i.e., there
are no azimuthal components of current.

Figure~\eqref{fig1} gives the geometry of a typical, arbitrarily oriented
wire. Assume that the wire is straight, even though the theory applies
equally to bent configurations. The same wire is also shown broken into
segments or subsections.

In equations \eqref{eq1}, \eqref{eq2}, and \eqref{eq3} below, the vector and
scalar potentials are given by

\begin{equation}
\vect{A} = \frac{\mu}{4\pi} \int_c I(s)\hat{s}(s)k(s-s^\prime)\dd{s}
\label{eq1}
\end{equation}

\begin{equation}
\Phi = \frac{1}{4\pi\varepsilon} \int_c q(s)k(s-s^\prime)\dd{s}
\label{eq2}
\end{equation}

where
\[ k(s-s\prime) = \frac{1}{2\pi}\int_{-\pi}^{\pi} \frac{e^{-jkr}}{r}\dd{\Phi}
\]

\[ r = ((s-s^\prime)^2 + 4a^2\sin^2\frac{\Phi}{2})^{\frac{1}{2}}
\]

and the linear charge density (via the continuity equation) is
\begin{equation}
q(s) = \frac{-1}{j\omega}\frac{\dd{I}}{\dd{s}}
\label{eq3}
\end{equation}

The kernel $k$ becomes the ``exact kernel'' when
$\vect{r}\rightarrow\vect{r^\prime}$ on $c$, but can be accurately
replaced by the ``reduced kernel,''
$k_0 = e^{-jkr}/r$, $|\vect{r} - \vect{r}(s)|^2+a^2(s)^{1/2}$
for $|\vect{r}-\vect{r^\prime}| \gg a$.

The integral equation relating the incident field, $\Einc$, and the
vector and scalar potentials is

\begin{equation}
-\Einc\cdot\hat{s}\times
-j\omega\hat{A}\cdot\hat{s}-\hat{s}\cdot\nabla\Phi
\quad .
\label{eq4}
\end{equation}

Equation \eqref{eq4}, above, is solved in MININEC by using the following
procedure.

The wires are divided into equal segments, and, as shown in
Figure~\ref{fig1}, the vectors $\vect{r}_n, n=0, 1, \ldots N+1$ are
defined, with respect to the global coordinate origin, $0^n$. The unit
vectors parallel to the wire axis for each segment shown are defined as

\begin{equation}
\hat{s}_{n+1/2} = \frac{\vect{r}_{n+1}
- \vect{r}_n}{|\vect{r}_{n+1}-\vect{r}_n|}
\quad .
\label{eq5}
\end{equation}

Pulse testing and pulse expansion functions used in MININEC are defined as
\begin{equation}
P_n(s) = \left\{
\begin{array}{ll}
1, & s_{n-1/2} < s < s_{n+1/2} \\
0, & \mathrm{otherwise}        \\
\end{array}\right.
\label{eq6}
\end{equation}

where the points $s_{n\pm1/2}$ designate segment midpoints,

\begin{equation}
s_{n+1/2} = \frac{s_{n+1} + s_n}{2}
\label{eq7}
\end{equation}

or in terms of the global coordinates,

\begin{equation}
\hat{r}_{n+1/2} = \frac{\hat{r}_{n+1} + \hat{r}_n}{2}
\quad.
\label{eq8}
\end{equation}

It is assumed that the components of the vectors $\Einc$
and $\vect{A}$ in equation \eqref{eq4} are sufficiently smooth over each
segment that their respective values on each segment may be replaced by
those taken at the point $s_m$. The pulse functions of \eqref{eq6} are
then used as testing functions on \eqref{eq4}, resulting in

\begin{equation}
\begin{aligned}
\Einc(s_m)&\cdot
\left[\left(\frac{s_m-s_{m-1}}{2}\right)\hat{s}_{m-1/2}
+ \left(\frac{s_{m+1} - s_m}{2}\right)\hat{s}_{m+1/2}\right] = \\
j\omega\vect{A}(s_m)&\cdot
\left[\left(\frac{s_m-s_{m-1}}{2}\right)\hat{s}_{m-1/2}
+ \left(\frac{s_{m+1}-s_m}{2}\right)\hat{s}_{m+1/2}\right] + \\
\Phi(s_{m+1/2}) &- \Phi(s_{m-1/2}) \\
\end{aligned}
\label{eq9}
\end{equation}

The vector quantities in brackets are simply
$(\vect{r}_{m+1/2} - \vect{r}_{m-1/2})$,
so \eqref{eq9} can be written as

\begin{equation}
\begin{gathered}
\Einc(s_m)\cdot(\vect{r}_{m+1/2}-\vect{r}_{m-1/2}) = \\
j\omega\vect{A}(s_m)\cdot(\vect{r}_{m+1/2}-\vect{r}_{m-1/2})
+\Phi(s_{m+1/2}) - \Phi(s_{m-1/2})
\quad.
\end{gathered}
\label{eq10}
\end{equation}

The currents are expanded in pulses centered at the junctions of
adjacent segments as illustrated in Figure~\ref{fig2a}. Note that pulses
are omitted from the wire ends. This is equivalent to placing a half
pulse of zero amplitude at each end, thus imposing the boundary
condition for zero current at unattached wire ends. The current
expansion can be written as

\begin{equation}
I(s) = \sum_{n=1}^{N} I_n P_n (s)
\quad.
\label{eq11}
\end{equation}

A difference approximation is applied to equation \eqref{eq3} to compute
the charge. Thus, as shown in Figure~\ref{fig2b}, the charge can be
represented as pulses displaced from the current pulses by a half pulse
width.

Substituting \eqref{eq11} into \eqref{eq10} produces a system of
equations that can be expressed in matrix form. Each matrix element,
$Z_{mn}$, associated with the $n$-th current and the $s_m$ observation
point involves scalar and vector potential terms with integrals of the
form

\begin{equation}
\Phi_{m,u,v} = \int_{s_u}^{s_v} k(s_m - s^\prime)\dd{s^\prime}
\label{eq12}
\end{equation}

where

\begin{equation}
k(s-s^\prime) = \frac{1}{2\pi}\int_{-\pi}^\pi\frac{e^{-jkr}m}{r_m}\dd{\Phi}
\label{eq13}
\end{equation}

and

\begin{equation}
r_m = \left((s_m - s^\prime)^2 + 4a^2\sin^2\frac{\Phi}{2}\right)^{1/2}
\quad.
\label{eq14}
\end{equation}

Equation \eqref{eq12} does not lend itself to straightforward
integration because of the singularity at $r=0$. The $1/r$ can be
subtracted from the integrand and then added as a separate term to yield

\begin{equation}
k(s-s^\prime) = \frac{1}{2\pi}\int_{-\pi}^\pi \frac{\dd{\Phi}}{r_m}
+\frac{1}{2\pi}\int_{-\pi}^\pi\frac{e^{-jkr}m-1}{r_m}\dd{\Phi}
\quad.
\label{eq15}
\end{equation}

The first term of \eqref{eq15} can be rewritten as an elliptic integral
of the first kind (reference~\cite{r6}).

\begin{equation}
\frac{\beta}{\pi a}F\left(\frac{\pi}{2}, \beta\right) =
\frac{1}{2\pi}\int_{-\pi}^\pi\frac{\dd{\Phi}}{r_m}
\label{eq16}
\end{equation}

where
\[ \beta=\frac{2a}{\left(s_m-s^\prime)^2+4a^2[\right]^{1/2}}
\]

$F(\frac{\pi}{2}$ has an approximation (reference~\ref{r6}).

\begin{equation}
\begin{aligned}
F\left(\frac{\pi}{2}, \beta\right) \cong\ 
& [a_0m + a_1m + a_2m^2 + a_3m^3] \cdot     \\
& [b_0 + b1_m + b_2m^2 + b_3m^3]\ln(1/m)    \\
\end{aligned}
\label{eq17}
\end{equation}

where

\[
m = 1 - \beta^2 = \frac{(s_m - s^\prime)^2}{(s_m - s^\prime)^2 + 4a^2}
\]

and

\[
\begin{array}{lrl}
a_0 = & 1.38629\ 436112 & b_0 = .5            \\
a_1 = &  .09666\ 344259 & b_1 = .12498\ 59397  \\
a_2 = &  .03590\ 092383 & b_2 = .06880\ 248576 \\
a_3 = &  .03742\ 563713 & b_3 = .03328\ 355346 \\
a_4 = &  .01451\ 196212 & b_4 = .00441\ 787012 \\
\end{array}
\]

Thus

\begin{equation}
\frac{\beta}{\pi a}F\left(\frac{\pi}{2},\beta\right)
\overrightarrow{s\rightarrow s^\prime}
-\frac{1}{\pi a}\ln\left[\frac{|s_m - s^\prime|}{8a}\right]
\label{eq18}
\end{equation}

and this singularity is also subtracted from $k(s_m-s^\prime)$.

Thus

\begin{equation}
\begin{gathered}
k(s_m - s^\prime) = -\frac{1}{\pi a}\ln \left[\frac{|s_m - s^\prime|}{8a}\right]
+\frac{\beta F(\frac{\pi}{2}, \beta)
      + \ln\left[\frac{|s_m - s^\prime|}{8a}\right]}{\pi a}  \\
+ \frac{1}{2\pi}\int_{-\pi}^\pi\frac{e^{-jkr}-1}{r}\dd{\Phi} \\
\end{gathered}
\label{eq19}
\end{equation}

This equation is substituted into equation~\eqref{eq12} and written as

\begin{equation}
\int_{s_u}^{s_v} k(s-s^\prime)\dd{s^\prime} = I_1 + I_2 + I_3
\quad.
\label{eq20}
\end{equation}

$I_1$, $I_2$, and $I_3$ are defined as

\begin{equation}
\begin{aligned}
I_1 & = -\frac{1}{\pi a}\int_{s_u}^{s_v}
        \ln\left[\frac{|s-s^\prime|}{8a}\right] \dd{s^\prime}
    & = \left.\frac{8}{\pi} u(1-\ln|u|)\right|_{u_1}^{u_2}
\end{aligned}
\label{eq21}
\end{equation}

where

\[ u_1 = \frac{s_u - s}{8a} \mbox{and} u_2 = \frac{s_v - s}{8a}
\]

Similarly,

\begin{equation}
I_2 = \int_{s_u}^{s_v}
\frac{\beta F\left(\frac{\pi}{2}, \beta\right)
+ \ln\frac{|s-s^\prime|}{8a}}{\pi a}\dd{s^\prime}
\label{eq22}
\end{equation}

This integral has a well behaved integrand and can be integrated
numerically. The integration is broken up into two integrals over the
ranges $(s_u, s)$ and $(s, s_v)$ for best accuracy. Gaussian quadrature
is used for the numerical integration (reference~\cite{r7}). The number
of points used in the integration routine is automatically selected by
consideration of the pulse accuracy required for the source to
observation distance. The final integral is

\begin{equation}
I_3 = \frac{1}{2\pi}\int_{s_u}^{s_v}\int_{-\pi}^\pi\frac{e^{-jkr}-1}{r}\dd{\Phi}
\quad.
\label{eq23}
\end{equation}

The integrand is nonsingular and can be integrated numerically. To
obviate the need for double integration, it is convenient to approximate
the integral by replacing $r$ by a reduced kernel approximation of
equation~\eqref{eq14}.

Thus

\begin{equation}
I_3 = \int_{s_u}^{s_v}\frac{e^{-jkr_a}-1}{r_a}\dd{s^\prime}
\label{eq24}
\end{equation}

where

\[
r_a = \sqrt{(s_v - s^\prime) + a^2}
\quad.
\]

The integral can be integrated numerically by the same procedure as for
$I_2$.

Thus, equation~\eqref{eq12} with its singularity problem is evaluated by
adding $I_1$ of equation~\eqref{eq21}, $I_2$ of equation~\eqref{eq22}, and
$I_3$ of equation~\eqref{eq24}.

This approach to evaluate~\eqref{eq12} is accurate for a wide range of
wire radii but breaks down when the radius becomes very small. For very
small radii, equation~\eqref{eq12} may be expressed as a single integral
and evaluated using two terms of a Maclaurin series, after Harrington
(reference~\cite{r5}). This approximation for the $\Phi$ terms is:

\begin{equation}
\begin{array}{ll}
\Phi = \frac{1}{2\pi\delta s}\ln\left(\frac{\delta s}{a}\right)
-j\frac{k}{4\pi}                   & \mbox{for } m=n          \\
\Phi = \frac{e^{-jkr_m}}{4\pi r_m} & \mbox{for } m\ne n \quad.\\
\end{array}
\label{eq25}
\end{equation}

Figure~\ref{fig3} demonstrates the range and validity with and without
the small radius correction. Without the correction, MININEC gives
acceptable answers for wire radii between $10^{-2}$ and $10^{-5}$ wave
lengths. Note that MININEC is within 10\% or better of the data
published by King (references~\cite{r8} and~\cite{r9}), for radii
between $10^{-3}$ and $10^{-2}$ wave lengths. The small radius
correction provides correct results for radii of $10^{-4}$ wave lengths
or smaller. In MININEC, the switch to the small radius approximations
occurs automatically for radii of $10^{-4}$ and smaller.

By substitution, the matrix equation to be solved is

\begin{equation}
[Z_{mn}] = [I_n] = [V_m]
\label{eq26}
\end{equation}

where

\begin{equation}
\begin{gathered}
[Z_{mn}] = \left.\frac{-1}{4\pi j\omega\varepsilon}\right[
k^2(\vect{r}_{m+1/2} - \vect{r}_{m-1/2}) \cdot
(\hat{s}_{n+1/2}\Phi_{m,n,n+1/2} + \hat{s}_{n-1/2}\Phi_{m,n-1/2,n}) \\
\left.
-\frac{\Phi_{m+1/2,n,n+1}}{s_{n+1}-s_n}+\frac{\Phi_{m+1/2,n-1,n}}{s_n-s_{n-1}}
+\frac{\Phi_{m+1/2,n,n+1}}{s_{n+1}-s_n}-\frac{\Phi_{m+1/2,n-1,n}}{s_n-s_{n-1}}
\right]\\
\end{gathered}
\label{eq27}
\end{equation}

and
\begin{equation}
V_m = \Einc(s_m)\cdot(\vect{r}_{m+1/2}-\vect{r}_{m-1/2})
\quad.
\label{eq28}
\end{equation}

$[Z_{mn}]$ is a square matrix and $[I_n]$ and $[V_m]$ are column
matrices with $n=1,2,\ldots N$ and $m=1,2,\cdot N$ total unknowns ($N$ is
the total number of current pulses). The extension of these equations to
two or more coupled wires follows the same line of development and will
not be covered here.

The column vector $[V_m]$ represents an applied voltage that
superimposes a constant tangential electric field along the wire for a
distance of one segment length centered conincident with the location of
the current pulses. Hence, for a transmitting antenna, all elements of
$V_m]$ are set to zero except for the element(s) corresponding to the
segment(s) located at the desired feed point(s). For an incident plane
wave, all elements of $[V_m]$ must be assigned a value depending on the
strength, polarization (or orientation), and angle of incidences of the
plane wave. The applied voltage source (transmit case), however, is the
only ready-made, or programmed, option in MININEC.

As stated above, the $[Z_{mn}]$ matrix in equation~\eqref{eq26} is
filled by the evaluation of an elliptic integral and use of Gaussian
quadrature for numerical integration. The solution of~\eqref{eq26} can
be accomplished by using any one of a number of standard matrix solution
techniques. MININEC(3) uses a triangular decomposition (LU
decomposition) with the Gauss elimination procedure with partial
pivoting (reference~\cite{r7}).

\section{Wire Junctions}
The theory developed thus far for straight wires is equally applicable
to bent wires. However, for coding simplicity in MININEC, bent wires are
treated in the same way as the junctions of multiple numbers of wires.
That is, a bend in an otherwise straight wire is treated as the junction
between two straight wires.

It has been generally accepted that the currents at junctions of thin
wires conform to Kirchoff's current law (reference~\cite{r10}). Rather
than explicitly enforcing this condition in MININEC, an overlapping
segment scheme (reference~\cite{r11}) is employed at junctions of two or
more wires. A detailed discussion of this approach, including arguments
for validity, appears in both references~\cite{r8} and~\cite{r9}. Only
those aspects essential to the use of MININEC are discussed here.

Consider a wire having no connections at either end. The wire is
subdivided into segments and the current is expanded in pulses centered
at adjacent segment junctions as described above and illustrated in
Figure~\ref{fig4a}. The end points hav no pulses, or alternatively the
end points have half pulses with zero amplitude. A second wire is to be
attached to one end of the first. The second wire is subdivided into
segments with pulses for currents located as in the first case. However,
a full pulse is located at the attachment end, with half the pulse
extending onto wire two, and half onto wire one, as illustrated in
Figure~\ref{fig4b}. The half on wire one assumes the dimensions (length
and radius) of the half segment on wire one, while the half on wire two
assumes the dimensions appropriate to wire two. Wire two overlaps onto
wire one with a full pulse centered at the junction end. Note that the
free end of the wire has a zero half pulse. A third wire may be assumed
to also overlap onto wire one, as illustrated in Figure~\ref{4c}. It can
be shown (see references~\cite{r8} and~\cite{r9}) that for a junction of
$N$ wires, only $N-1$ overlapping pulses are required to satisfy
Kirchoff's current law. Alternatively, wire three could have overlapped
onto wire two (not illustrated here).

The convention in MININEC(3) is that the overlap occurs onto the
earliest wire specified at a given junction. It is assumed that a wire
can overlap onto another wire, provided that another wire was previously
specified. It cannot overlap onto a wire not yet specified. Either end
of a wire may overlap onto either end of another wire. All that is
required to impose the continuity conditions at the junction is that
there be $N-1$ overlaps for a junction of $N$ wires.

Current reference directions are assumed to be based on the order in
which the coordinates of a wire are specified. A positive wire current
is from the end first specified, end one, towards the other end, end
two. By use of Kirchoff's current law and the current reference
direction, the currents at the junction can be found. For example,
suppose the wires in Figure~\ref{fig4c} are all specified from left to
right. Let the pulse amplitudes for the first pulse on wires two and
three be $I_2$ and $I_3$, respectively. Then the currents out of the
junctions into wires two and three are the complex amplitudes of the
first pules, the overlapping pulses, on wires two and three,
respectively. Hence, the current on wire one into the junction is the
sum of these currents; i.e., $I_1=I_2+I_3$.

MININEC(3) automatically determines, during geometry input, whether
there is a connection on either end of a wire, and if so, to which wire
and wire end it is connected. After solving for the current pulse
amplitudes, MININEC(3) then computes the junction currents, if any, for
each wire end. The final display indicates free ends by the letter
\verb+E+ (for free end) and junction ends by the letter \verb+J+. The
geometry and currents are displayed wire by wire.

\subsection{The Ground Plane}
\label{sec-groundplane}

The method of images is used in MININEC to solve for currents in wires
located over a perfectly conducting ground plane.

Consider a wire structure represented by $N$ segments. In the presence
of a perfectly conducting ground plane, by image theory, the structure
and ground plane may be replaced by the original structure and its
image. Hence, there are now $2N$ segments and $2N$ unknowns to be
determined. Equation~\eqref{eq26} can be written as

\begin{equation}
\left[
\begin{array}{l}
V_1    \\
\vdots \\
V_N    \\
\vdots \\
V_{2N} \\
\end{array}
\right]
=
\left[
\begin{array}{lllll}
Z_{11}   & \cdots & Z_{1N} & \cdots & Z_{1,2N}  \\
\vdots   &        &\vdots  &        & \vdots    \\
Z_{N1}   & \cdots & Z_{NN} & \cdots & Z_{N,2N}  \\
\vdots   &        &\vdots  &        & \vdots    \\
Z_{2N,1} & \cdots &\cdot   & \cdots & Z_{2N,2N} \\
\end{array}
\right]
=
\left[
\begin{array}{l}
I_1    \\
\vdots \\
I_N    \\
\vdots \\
I_{2N}    \\
\end{array}
\right]
\label{eq29}
\end{equation}

The image current, $I_{N+1}\cdots I_{2N}$, are equal to the currents on
the original structure, $I_1\cdots I_N$, so that $I_n = I_{2N-n+1}$. Half
the equations represented in~\eqref{eq29} contain redundant information
and may be discarded. It may be reduced to a square matrix again by
adding appropriate columns; i.e., by using the current identity.
Hence~\eqref{eq29} becomes

\begin{equation}
V=[Z^\prime_{ij}I
\label{eq30}
\end{equation}

where $Z^\prime_{ij} = Z_{ij} + Z_{i\ 2N-j+1}$.

For a wire attached to ground, a current pulse is automatically added to
the wire end point connected to ground so that current continuity with
its image is observed; i.e., a non-zero half pulse is placed on both the
wire end and its image. The voltage in equation~\eqref{eq30} is divided
by two in this case. Either end of a wire may be attached to ground.

\subsection{Lumped Parameter Loading}
The wire structures discussed so far consist of perfectly conducting
wires. If an impedance due to a fixed load, $Z_L = R + jX$, is added to
the structure so that its location coincides with that of one or more of
the non-zero-current pulse functions (i.e., a lumped load is placed on
the wire at the junction of two segments), then the load introduces an
additional voltage (a voltage drop) equal to the product of the current
pulse magnitude and $Z_L$. Hence, equation~\eqref{eq26} becomes

\begin{equation}
[Z^\prime_{mn}][I_n]=[V_m]
\label{eq31}
\end{equation}

where $Z^\prime_{mn}=Z_{mn}$ for $m\ne n$ and $Z^\prime_{mn}=Z_mn+Z_L$
for $m=n$. Hence, a specified impedance represented as the sum of a
resistance and a reactance, and located on a wire coincident with a
current pulse is simply added to the diagonal impedance element or
self-term corresponding to that pulse. A distributed impedance such as
wire conductivity can be treated in the same way by use of an
equivalent, lumped-circuit, element-impedance relationship.

\subsection{Near Fields}
The electric near fields and the magnetic near fields can be determined
from the current distribution obtained in the solution of
equation~\eqref{26}.

The near electric fields are computed by the method described by
A.~T.~Adams, et al., (reference~\cite{r12}). Using MININEC, the current
on the wire structure is approximated using the computed current pulses.
To determine the electric field at a given point in the near field, a
small, virtual thin-wire dipole is placed tat the point with its axis
parallel to the appropriate vector component. The open-circuit voltage
at the near field point can be calculated from the knowledge of the
current distribution over the wire structure and the mutual impedance
between the wire structure and the virtual dipole. In other words,

\begin{equation}
V_d = \sum_{i=1}^{N}Z_{di}I_i \quad.
\label{eq32}
\end{equation}

The virtual dipole is open-circuited. $V_d$ is the open-circuit voltage.
$I_i$ are the MININEC computed current pulses of the wire structure.
$Z_{di}$ are the mutual impedances between the wire structure and the
virtual dipole. The mutual impedances are calculated using the MININEC
method of equation~\eqref{eq27}. The electric field strength along the
direction of the virtual dipole is given by

\begin{equation}
E_d = -\frac{V_d}{\mbox{length of dipole}}\quad.
\label{eq33}
\end{equation}

This equation is evaluated once for each electric field vector component
in the $x$, $y$ and $z$ directions at the near field point of interest.
In MININEC, a virtual dipole of length .001 wave length is used.

MININEC calculates the three vector components $E_x$, $E_y$ and $E_z$,
as real and imaginary terms, from which the magnitude and phase are
determined. The average value is determined by

\begin{equation}
\ave{E} = \left[\frac{1}{2}(E_x^2+E_y^2+E_z^2)\right]^{1/2}
\label{eq34}
\end{equation}

which is a conservative estimate of the maximum value. The maximum or
peak electric field is determined by the method described by Adams and
Mendelovicz (reference~\cite{r13}). The peak electric field is

\begin{equation}
\peak{E} = \left[\frac{1}{2}(E_x^2+E_y^2+E_z^2)
                +\frac{1}{2}(A^2+B^2)^{1/2} \right]^{1/2}
\label{eq35}
\end{equation}

where
\[ A = E_x^2\cos 2\theta_x + E_y^2\cos 2\theta_y + E_z^2\cos 2\theta_z
\]
\[ B = E_x^2\sin 2\theta_x + E_y^2\sin 2\theta_y + E_z^2\sin 2\theta_z
\]

and where $O_x$, $O_y$, and $O_z$ are the phase angles for the
corresponding field component.

The near magnetic fields are computed by a comparable method. As is the
case for electric near fields, the currents on the wires are
approximated by the current pulses of the MININEC solution. A virtual,
thin-wire dipole is placed at the near field point with its axis
parallel to the appropriate vector component. The near magnetic field is
then calculated using the MININEC current distribution and the
difference between the appropriate components of the vector potential.

The vector potential is generally defined such that 

\begin{equation}
\Hplus = \frac{1}{\mu}\nabla X\Aplus
\label{eq36}
\end{equation}

expressed in rectangular coordinates, becomes

\begin{equation}
\mu\Hplus =
\left(\frac{\partial A_z}{\partial y} - \frac{\partial A_y}{\partial z}\right)
\hat{i} +
\left(\frac{\partial A_x}{\partial z} - \frac{\partial A_z}{\partial x}\right)
\hat{j}+
\left(\frac{\partial A_y}{\partial x} - \frac{\partial A_x}{\partial y}\right)
\hat{k}
\label{eq37}
\end{equation}

whre $\hat{i}$, $\hat{j}$, $\hat{k}$ are the unit vectors parallel to
the $x$, $y$, $z$ coordinate axis, respectively. And where $A_x$, $A_y$,
$A_z$ are the corresponding components of the vector potential evaluated
at the location of the virtual dipole (i.e., the near field point). If
the virtual dipoles are electrically short enough so that the fields
vary continuously and smoothly over the dipole length, the partial
derivatives of equation~\eqref{eq37} can be replaced by differences:

\begin{equation}
\mu\Hplus =
\left(\frac{\delta A_z}{\delta y} - \frac{\delta A_y}{\delta z}\right)
\hat{i} +
\left(\frac{\delta A_x}{\delta z} - \frac{\delta A_z}{\delta x}\right)
\hat{j}+
\left(\frac{\delta A_y}{\delta x} - \frac{\delta A_x}{\delta y}\right)
\hat{k}
\label{eq38}
\end{equation}

such that, for example, $\delta A_z/\delta_y$ is the change in the
Z-component of the vector potential along a y-directed virtual dipole of
length $\delta y$, located at the near field point, etc. In MININEC, the
virtual dipole length is .001 wave length for both the near electric and
near magnetic field calculations.

MININEC calculates the three vector components, $H_x$, $H_y$ and $H_z$
as real and imaginary terms, from which the magnitude and phase are
determined. The average and peak values of the magnetic near fields are
found in the same way as they are for the electric near fields. Thus,

\begin{equation}
\ave{H} = \left[\frac{1}{2}(H_x^2 + H_y^2 + H_z^2)\right]^{1/2}
\label{eq39}
\end{equation}

and

\begin{equation}
\peak{H} = \left[\frac{1}{2}(H_x^2 + H_y^2 + H_z^2)
                +\frac{1}{2}(A^2 + B^2)^{1/2} \right]^{1/2}
\label{eq40}
\end{equation}

where

\[ A = H_x^2\cos 2\theta_x + H_y^2\cos2\theta_y + H_z^2\cos2\theta_z
\]
\[ B = H_x^2\sin 2\theta_x + H_y^2\sin2\theta_y + H_z^2\sin2\theta_z
\]

and $\theta_i$ are the corresponding phase angles.

Both the electric and magnetic near fields can be scaled for any desired
radiated power from the wire structure since the near fields are
directly proportional to the square root of the power radiated.

\section{Far Zone Radiation Patterns}
Once the induced currents on the wires have been determined from
equation~\eqref{eq25}, the radiated electric fields are computed by

\begin{equation}
\vect{E}(\vect{r}_0) = \int_L\frac{jks}{\pi s} + \frac{e^{-jkr_0}}{r_0}
+[\hat{k} + \vect{I}(a)\hat{k}-\vect{I}(a)]e^{i\vect{K}\cdot\vect{r}}\dd{s}
\label{eq41}
\end{equation}

whre $\vect{r}_0$ is the position vector at the observation point,
$\bar{k} = \vect{r}_0 / |\vect{r}_0|$, and
$\vect{K} = k\hat{k} = \frac{2\pi}{\lambda}\hat{k}$. The integral is
evaluated in closed form over each straight wire segment for each
current pulse and is reduced to tha summation over the wire segments.
The fields are then evaluated as real and imaginary parts of the
$\theta$ and $\phi$ copmponents at a specified radial distance. If the
radial distance is zero, the factor $e^{-jkr_0}/r_0$ defaults to unity.

The power gain in MININEC is evaluated with the $e^{-jkr_0} / r_0$
factor set to one in an approach similar to that in NEC
(reference~\cite{r4}). The power gain in a given direction
$(\theta, \phi)$ in sperical coordinates is

\begin{equation}
G = 10 \log\left(\frac{4\pi P(\theta,\phi)}{\Pin}\right)
\label{eq42}
\end{equation}

where $P(\theta,\phi)$ is the power radiated per unit steradian in the
direction $(\theta, \phi)$ and $\Pin$ is the total input power to the
antenna. Note that directive gain could be obtained by replacing
$\Pin$ by the total power radiated. This step is not done in MININEC.
$\Pin$ is calculated from the applied voltages and the corresponding
feed point currents as

\begin{equation}
\Pin = \sum_{i=1}^{N}(1/2) \Re(V_n I_n^*)
\label{eq43}
\end{equation}

where $I_n^*$ denotes complex conjugate and $n$ is the number of
sources. $P(\theta,\phi)$ is determined from

\begin{equation}
\P(\theta,\phi) = 1/2 r_0^2 \Re[\vect{E}\times\vect{H}
= \frac{r_0^2}{2n}\vect{E}\cdot\vect{E}^*
\label{eq44}
\end{equation}

where $r_0$ is the magnitude of the position vector $\vect{r}_0$ in the
$(\theta,\phi)$ direction. In MININEC, the gains are calculated for the
individual orthogonal components of the field determined from
equation~\eqref{eq41}. The power gain thus obtained from \eqref{eq42} is
in dB above the gain of an isotropic antenna (sometimes denoted as dBi).

MININEC includes an option to correct the far fields and gain for the
effects of real ground using a Fresnel reflection coefficient. The
method is similar to the far field corrections used in NEC
(reference~\cite{r4}), but is not limited to one or two mediums. The
surface of the ground is divided into a finite number of zones with a
constant conductivity and dielectric constant in each zone, i.e., a
constant surface impedance for each zone. The zones are defined by
circular boundaries concentric about the origin or linear boundaries
parallel to the y-axis and spaced along the positive x-axis. Thus, in
the latter case, the ground surface is divided into ``strips'' at user
defined x-axis intercepts. In the former case, the ground surface is
divided into concentric rings at user specified radii. In this case, the
first ring, or zone, may include a radial wire ground screen. For both
circular and linear zone grounds, each zone may have a different surface
impedance and each zone may have a different height (Z-coordinate)
relative to the first zone. In MININEC, the number of zones is limited
by an array dimension and is currently set to 5.

In the Fresnel reflection coefficient method, the far field is obtained
by summing the contributions of a direct ray and a reflected ray from
each current pulse. The field, due to the reflected ray, is modified by
the Fresnel plane wave reflection coefficient, which depends on the
ground surface impedance at the bounce point, or specular point, and the
angle of incidence.

The Fresnel reflection coefficients have not been applied to the MININEC
current calculation. When a real ground is specified, the currents are
calculated by using the perfectly conducting image theory (described in
section~\ref{sec-groundplane}. The real ground corrections are applied
to the far field calculations only. This compromise is designed to keep
MININEC relatively compact and provide accurate results whenever the
ground directly beneath the antenna is a good conductor.

Following along the lines of the development given by Burke and Poggio
(reference~\cite{r4}), a wave incident upon a finite ground (i.e., a
real ground) yields a reflected field, $\vect{E}_r$, given by

\begin{equation}
\vect{E}_R = R_H\left[(\vect{E}_I\cdot\hat{P})\hat{P}\right]
           + R_V\left[\vect{E}_I-(\vect{E}_I\cdot\hat{P})\hat{P}\right]
\label{eq45}
\end{equation}

or

\begin{equation}
\vect{E}_R = R_V\vect{E}_I + (R_H - R_V)(\vect{E}_I\cdot\hat{P})\hat{P}
\label{eq46}
\end{equation}

where $\hat{P}$ is a unit vector perpendicular to the plane of
incidence. $\vect{E}_I$ is the incident field and $R_V$ and $R_H$ are
the vertical and horizontal reflection coefficients, respectively.

The two terms in square brackets in equation~\eqref{eq44} correspond to
horizontally and vertically polarized waves. The reflected field is
obtained by decomposing the incident field into horizontally and
vertically polarized waves, computing a reflected wave for each, and
recombining the two.

The vertical and horizontal coefficients are

\begin{equation}
R_V = \frac{\cos\theta - z\sqrt{1-z^2\sin^2\theta}}
           {\cos\theta + z\sqrt{1-z^2\sin^2\theta}}
\label{eq47}
\end{equation}

\begin{equation}
R_H = \frac{-z\cos\theta - \sqrt{1-z^2\sin^2\theta}}
           {\cos\theta + z\sqrt{1-z^2\sin^2\theta}}
\label{eq48}
\end{equation}

where $\theta$ is the angle of incidence and $Z$ is the relative
impedance of the ground surface (relative to the free space impedance).

For a given observation direction $(\theta, \phi)$, the $\hat{P}$ vector
normal to the plane of incidence is

\begin{equation}
\hat{P} = (-\sin\theta, \cos\phi)
\label{eq49}
\end{equation}

as may be seen in Figure~\ref{fig5c}. In addition, the $\vect{r}_0$
vector, pointing in the observation direction, is

\begin{equation}
\vect{r}_0 = (\sin\theta\cos\phi, sin\theta\sin\phi, \cos\theta)
\label{eq50}
\end{equation}

To obtain the far fields, the integral in equation~\eqref{eq40} implies
the summation over all the current pulses. For the direct field, the
currents and vectors pointing to the current pulse centers are
calculated and stored in arrays by MININEC during the matrix solution
process. The incident field on the ground surface is also computed from
a summation over the currents, but this requires the coordinates of the
specular point (the bounce point). For the geometry illustrated in
Figure~\ref{fig5}, the specular point is given by

\begin{equation}
\bounce{r} = \sqrt{\bounce{x}^2 + \bounce{y}^2}
\label{eq51}
\end{equation}

\begin{equation}
\bounce{x} = x_i + d\cos\phi
\label{eq52}
\end{equation}

\begin{equation}
\bounce{y} = y_i + d\sin\phi
\label{eq53}
\end{equation}

\begin{equation}
d = Z_i \tan\theta
\label{eq54}
\end{equation}

The value of $\bounce{x}$ or $\bounce{r}$ is used
appropriately for the case of linear or circular zone boundaries to
determine in which media the bounce occurs. The height of the ground at
this point is used to locate the image of the source.

The ground surface impedance in any zone is given by

\begin{equation}
Z_g = \frac{1}{\sqrt{\frac{\varepsilon}{\varepsilon_0}
                    - j\frac{G}{\omega\varepsilon_0}}}
\label{eq55}
\end{equation}

The surface impedance, when a ground screen is present and the specular
point lies on the ground screen, is given by Wait in
reference~\cite{r14}, (also see reference~\cite{r4}). The impedance of
the ground screen by itself is

\[
Z_gs (\bounce{r} = ?\sqrt{\varepsilon_0\mu_0}
\frac{?\bounce{r}}{N} ?? \frac{\bounce{r}}{NC_0}
\]

where $N$ is the number of wires in the ground screen and $C_0$ is the
radius of each wire. The effective ground impedance is formed by
computing the parallel impedance of the ground without the ground screen
and the impedance of the ground screen without the ground, or

\begin{equation}
Z = \frac{Z_g Z_{gs}}{Z_g + Z_{gs}}
\label{eq56}
\end{equation}

(where $Z=Z_g$ if no ground screen is present).

The total field at a point $(\theta,\phi,r)$ is the vector sum of the
direct and reflected fields as described. When the range $r$
is set to zero or the power gain option is selected, the
$e^{-jkr} / r$ term is set to unity. The total resulting field is used
in equation~\eqref{eq43} to calculate the power gain.

\section{Validation and Modeling Guidance}
The solution to an antenna problem generated by a method of moments
computer program is, at best, an approximation. How close the solution
is to reality depends in part on (1) the numerical methods employed in
the code (and how well these methods are implemented), (2) the inherent
accuracy (i.e., the number of significant digits) of the computer, (3)
how well the antenna being modeled conforms to the limitations (i.e.,
simplifying assumptions) of the electromagnetics formulation used to
create the computer program, and (4) the user's experience. Nonetheless,
highly acurate answers can be obtained by careful modeling of the
antenna configuration, taking into account the inherent limitations of
the computer program.

Reliable, accurate answers are obtained when the user has accumulated
sufficient experience from frequent and systematic exercise of the
program to recognize problem areas. He must be fully awarte of potential
difficulties throughout the modeling process, from initially setting up
a problem to interpreting the results. It is recommended that the user
run MININEC for a number of elementary problems, comparing the results
to independent solutions or real world measurements, until he has the
confidence to apply the code to a problem for which the answer is
unknown. The examples in this section provide a good place to start.

Development of confidence in the computer solution is a process of
discovery of the limitations of the computer program. This entails the
modeling of a number of simple antenna structures found in standard
texts on antenna theory. The results may be used as guidelines to model
more complex antenna geometries. Natural first choices are dipoles in
free space and over ground (or monopoles), followed by TEE and inverted
L shaped antennas, etc. For each antenna type, a number of problems are
selected involving different wire lengths and radii. MININEC is run for
each angenna problem a number of times, varying segmentation (i.e., a
convergence test) and other parameters to reveal the program
limitations. Insight for effective application of the program is gained
from comparisons with measured data or analytical solutions (when
available). In this manner, modeling guidelines are derived and updated
from simpler antenna problems for which there is reliable measured data
or generally accepted theoretical data.

\subsection{Dipole Antennas}
The theoretical behavior of the dipole antenna has been studied
intensely, and the literature is rich with exammples that include
measured and theoretical data, references~\cite{r8},~\cite{r9},
\ref{r15},~\ref{r16}. Comparison of MININEC results to this data for
dipoles of various dimensions establishes validation and provides the
basis for modeling guidelines. For example, convergence tests for
various length dipoles reveals the accuracy that can be expected and
provides a rational criterion for selection of segmentation density (the
number of segments per wire) based on wire length in wave lengths.

Figure~\ref{fig6} through \ref{fig11} show the results of convergence
tests for an electrically short dipole (much shorter than the first
resonance length), a dipole near resonance, and a dipole near
antiresonance, respectively. Each dipole is electrically thin and center
driven. Part (a) Figures (\ref{fig6},~\ref{fig8}, and~\ref{fig10}) give
the variation of admittance, versus the number of segments. Part (b)
Figures (\ref{fig7},~\ref{fig9}, and~\ref{fig11}) give the percent
difference between MININEC and the values published by R.W.P. King,
references~\cite{r8} and~\cite{r9}.

The electrically short dipole, Figures~\ref{fig6} and~\ref{fig7}, and
the half wave dipole, Figures~\ref{fig8} and~\ref{fig9}, show definitive
signs of convergence and stability. No sign of convergence is seen for
the antiresonant dipole, Figure~\ref{fig10} and~\ref{fig11}. The authors
also have seen similar convergence problems near antiresonance for other
method of moment codes (notably NEC). Figures~\ref{fig6}
through~\ref{fig11} can be used as guidance for selection of the
segmentation density. An antenna is modeled as a collection of wires.
Each wire is divided into a number of short segments selected by the
user. The number of segments to achieve a desirable confidence level can
be based on the results of Figures~\ref{fig6} through~\ref{fig11},
depending on the wire length in wave lengths. Using this data does not
guarantee convergence or the percent accuracy for a more complicated
antenna, but it does provide a starting point. Convergence testing is
always advisable.

Given the convergence properties of MININEC, how well does it predict
dipole properties? Figure~\ref{fig12} is a comparison of a single,
30-segment MININEC model to the theory given by King
(references~\cite{r9}). Shown is the admittance-versus-dipole length for
both MININEC and King. The difference is less that .5 millimho for most
of the range, with the greatest difference of about 1.5 millimhos in the
susceptance at a dipole length of .64 wave lengths. For longer or
shorter antennas, the user is advised to perform suitable convergence
tests.

The accuracy of the method of moments solution depends also on meeting
the thin wire criterion. To illustrate, Figure~\ref{fig13} shows the
variation of admittance versus the wire radius. The data given by King
(references~\cite{r9}) are slso shown for radii betwenn $10^{-3}$ and
$10^{-2}$ wave lengths. The segment to radius ratio, $\delta/a$, is 25
at $^01{-3}$ and 2.5 at $10^{-2}$. For thicker wires than $10^{-2}$, the
thin wire criteria is not achieved and the results are expected to be
not as good. The data show valid behavior for thin wires with
$\delta/a > 2.5$ or radii of $10^{-2}$ wave lengths and smaller.

Numerical problems may occur in the solution when quantities become too
small for the inherent accuracy of the computer. An example is the
erroneous results that can occur for very short segments.
Figure~\ref{fig14} shows the results of a test designed to identify the
sort segment limit. Shown is the admittance-versus-dipole length in wave
lengths for a 10-segment dipole in free space. The conductance and
susceptance displays the proper behavior for a dipole length greater
than $10^{-3}$ wave lengths. This corresponds to a segment length of
$10^{-4}$ wave lengths and longer. Below $10^{-3}$, the conductance
oscillates about the expected values as the segment length is reduced,
and at times displays negative, non-physical values. A change from
single precision to double precision extends the validity range to even
shorter segments, but significantly increases the solution time beyond
acceptable run times for a 16-bit microcomputer. For MININEC, on a
16-bit machine in single precision, the segment length should always be
greater than $10^{-4}$ wave lengths.

Antennas are often constructed of wires and towers or other conductors
with vastly different radii. Even simple dipoles may have tapered
elements. A typical MININEC model may therefore involve the connection
of wires with large step changes in radii. The stepped-radius wire
junction has been extensively studied by Glisson and Wilton
(references~\cite{r17}). Figure~\ref{fig15} is the stepped wire geometry
used in their study. They adapted a body of revolution computer code,
PEC; to solve very accurately for currents and charges along the stepped
wire antenna. The results were compared to NEC in this study.

Figures~\ref{fig16},~\ref{fig17},~\ref{fig18}, and~\ref{fig19} show the
current distribution predicted by PEC, NEC and MININEC for radii steps
of 1:1.25, 1:5, 1:10 and 1:100, respectively. The NEC data are from
their report. The MININEC results follow the PEC data surprisingly well
for all step ratios. (We believe the difference between NEC and PEC may
be an error in the data in the Glisson report. We have not observed this
difference in NEC data.) Further investigation of MININEC for different
stepped radius problems should be conducted. Suggestions include moving
the feed closer to the step and switching the radii $a_1$ and $a_2$.

Multiple wire antenna structures may often require very close spacing.
When the spacing is very small, the currents may not be adequately
represented by a thin filament on the wire axis as it is represented in
MININEC. Figures~\ref{fig20},~\ref{21},~\ref{fig22}, and~\ref{fig23}
show MININEC data for a parallel wire test used to investigate the close
spacing limit. The test consists of evaluating the self and mutual
admittance between two parallel half wave dipoles. One antenna is driven
(i.e., the source is in the center pulse) while the second is not. The
self admittance is the feed point current on the first wire if the
applied voltage is 1 + j0 volts and the mutual admittance is the current
for the center pulse of the second wire.

Figure~\ref{fig20} shows the self admittance compared to the theory by
R.W.P. King (references~\cite{r9}) for dipole center to center spacings
between .1 and .5 wave length. Figure~\ref{fig21} shows a similar
comparison for the mutual admittance over the same range. The
differences between MININEC and R.W.P. King are mostly less than .2
millimho and are no greater than .4 millimho in the worst case over the
range shown for both self and mutual admittance.

Given the good agreement with theory down to a spacing of .1 wave
length, how does MININEC fare for closer spacing? Figures~\ref{fig22}
and~\ref{fig23} show the magnitude and phase of the mutual admittance
for spacings down to the point of contact of the two parallel dipoles.
Keep in mind the good agreement between MININEC and theory for spacings
of .1 wave length and greater. (Reference~\cite{r9} does not provide
data for spacings less than .1, no comparison is shown.) It can be seen
that the magnitude and phase continue smoothly as the spacing is
reduced. Although these data are not conclusive, it can be implied that
MININEC can model antenna configurations with wire spacings less than .1
wave length. Whenever a model has close spacing, however, it is
advisable to examine the results very closely to ensure proper behavior.

\subsection{Loop Antennas}
A circular wire loop antenna may be modeled by connecting a number of
wires to form a polygon approximation to the circular loop. A simple
model has one segment per wire, with each wire forming one side of the
polygon model, so that the number of sides and the number of segments
are equal. For a given circumference, the number of wires, and hence the
number of segments, can be increased until the solution stabilizes,
indicating the number of seides required to model the circular loop.
Figure~\ref{fig24} shows the results of this procedure for a loop, one
wave length in circumference. The polygon model is circumscribed by a
circle whose circumference is one wave length. The wire radius
($a=.00674$ meter) is chosen to correspond to the published data given
by R.W.P. King (references~\cite{r9}). At best, the real part of the
MININEC admittance comes to within 3\% of King's data and the imaginary
part approaches to within 6\%. For 22 segments (and 22 sides) the
percent difference in real and imaginary is about equal, and less than
6\% for each.

Figure~\ref{fig25} compares MININEC and R.W.P. King admittance data for
a range of loop diameters from .1 to 2.0 wave lengths. The MININEC model
is the 22 segment or 22 sided polygon loop. The agreement is excellent.
The difference between King and MININEC is no greater than .4 millimho
over the entire range. From .1 to .8 wave length, the MININEC data and
King data are virtually identical.

Figures~\ref{fig26} and~\ref{fig27} show MININEC data for small loops
with a circumference from $10^{-3}$ to just above .4 wave length. Keep
in mind the excellent agreement with King's data for loops of .1 and
greater (Figure~\ref{fig25}). The real and imaginary parts of the
adminttance in Figures~\ref{fig26} and~\ref{fig27}, respectively, are
well behaved for loops greater than $10^{-2}$ wave lengths. Below
$10^{-2}$, the real part of the admittance becomes unstable due to
numerical problems encountered at the limits of single precision. Note
that at 22 segments, the segment size at a circumference of $10^{-3}$,
is very nearly the same short segment length limit displayed by the
dipole test in Figure~\ref{fig14}. The data in
Figures~\ref{fig25},~\ref{fig26}, and~\ref{fig27} suggest a small loop
limit for MININEC (on a 16-bit, single precision microcomputer) of
$10^{-2}$ wave lengths in circumference. This corresponds to a loop 18
inches in diameter at 2~MHz (about the size of a basketball goal).

\subsection{Monopoles and Antennas Above Ground}
Simply stated, an antenna above a perfectly conducting ground plane is
equivalent to the original antenna and its mirror image in free space.
Hence, all the modeling results and guidelines presented so far are
directly applicable to monopoles. Specifically, the convergence
properties illustrated in Figures~\ref{fig6} through~\ref{fig11} can be
used for the initial selection of the segmentation required for a
monopole. However, this should not preclude convergence testing whenever
possible.

Figure~\ref{fig28} illustrates the geometry of a TEE-antenna. The
antenna is driven or fed at its base from a coaxial termination at the
ground plane. The dimensions for two Tee-antenna designs ($K_0 h = .2$
and $K_0 h = .5$) are also given. A convergence test was performed for
each antenna using the segmentation scheme in the table. The results of
these tests are given in Figures~\ref{fig29} and~\ref{fig30} for
$K_0 h = .2$ and $K_0 h = .5$, respectively. A comparison of the
``best'' results to the measurements of Prasad and King
(references~\cite{r18}) for MININEC and several other codes is given in
Figure~\ref{fig31}.

Figure~\ref{fig31} compares five computer programs including MININEC for
the two TEE-antennas. In each case, the programs were tested for
convergence and the best answer with respect to Prasad's measurements is
given. NEC is the code previously described (references~\cite{r4}). TGP
(Triangular-Galerkin Procedure) is teh code written by Chao and Straight
(references~\cite{r11}) using triangular expansion and testing functions
in a Galerkin procedure (i.e., triangles for both testing and expansion
functions). PSRT (Piece-wise Sinusoidal Reaction Technique) is a
sinusoidal Galerkin code written by Richmond (references~\cite{r19}).
TWTD (Thin Wire Time Domain) is a time domain method of moments code
written by Van Blaricum and Miller (references~\cite{r20}). TWTD uses
subsection collocation (quadratic interpolation with point matching) to
solve for the time-dependent induced currents and time-dependent
radiated fields. Admittance data are obtained from a discrete Fourier
transform of the source current. All codes except MININEC are in FORTRAN
and require mainframe (large) computers. The data show that MININEC can
provide equally accurate answers.

\end{document}
